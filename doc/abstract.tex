% abstract.tex (Abstract)

%\addcontentsline{toc}{chapter}{Resumen}

\chapter*{Resumen}

La planificaci\'on preoperatoria es un importante paso que debe realizarse previo a un procedimiento quir\'urgico. Los sistemas de CAOS (\textit{Computer Aided Orthopedic Surgery}) son utilizados ampliamente en la planificaci\'on preoperatoria de cirug\'ias para fracturas de extremidades inferiores. Estos sistemas toman como entrada una imagen de Rayos-X y la planificaci\'on se realiza de forma digital. Como parte de la planificaci\'on se debe colocar todos los fragmentos de huesos originado por una fractura y colocarlos en su posici\'on anat\'omicamente correcta por parte del m\'edico tratante (proceso de reducci\'on de fractura). En muchos casos se requiere colocar un implante al paciente como parte del procedimiento de reducci\'on. Cuando el implante no se ajusta perfectamente a la anatom\'ia del paciente, \'este debe ser ajustado al hueso. En este trabajo se propone un esquema para la planificaci\'on preoperatoria de fracturas en los miembros inferiores a ser utilizado en PCs convencionales. El esquema contiene una serie de etapas con funcionalidades bien definidas para el desarrollo del proceso, una de estas etapas consiste en la deformaci\'on del implante a colocar. En este trabajo se presenta un nuevo m\'etodo para la deformaci\'on basado en el m\'etodo MLS (\textit{Moving Least Squares}) que ayuda al cirujano en su tarea de ejecutar la deformaci\'on del implante en el quir\'ofano. Diversas mejoras son introducidas con el objetivo de alcanzar resultados visuales muy similares al procedimiento real efectuado en el quir\'ofano. Los detalles de nuestra propuesta son explicados completamente as\'i como todos los par\'ametros provistos. Un total de m\'as de 100 casos cl\'inicos han sido planificados de manera exitosa en el Servicio de Traumatolog\'ia y Ortopedia del Hospital Universitario de Caracas \cite{REF_HUC} empleando nuestra propuesta.