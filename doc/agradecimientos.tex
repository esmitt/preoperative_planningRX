\chapter*{Agradecimientos}

Quiero agradecer al Big Boss por tomar el control de todo y estar siempre presente, por darme la oportunidad en estos a\~nos de consolidar mi familia de una manera espectacular.

Siempre agradezco a mis padres y mis t\'i@s por estar siempre presentes y creer en m\'i y ayudarme a conseguir mis metas propuestas. Gracias a mi pap� por siempre estar pendiente en el desarrollo de este trabajo. A mi hermana y mis \textit{locas} por estar siempre presente d\'andome sonrisas. A mi t\'ia Bena, que es mi otra mam\'a desde siempre. En general, a mi familia que representan una gran, fuerte y maravillosa fuente de apoyo y sabidur�a durante toda mi vida, particularmente gracias a mi t�a Chela y mi abuela.

Agradezco enormemente a mi compa\~nera de vida la cual ha estado presente en todo el desarrollo de este trabajo, a m\'i princesa. Gracias por todo t\'u apoyo y t\'u comprensi\'on. 

Quiero agradecer a mi tutor de tesis y de plan de formaci\'on, el Dr. Ernesto Coto, por sus conocimientos invaluables que me brindo para llevar a cabo esta investigaci\'on as\'i como mi formaci\'on, y sobretodo su gran paciencia para lograr resultados satisfactorios y finalizar exitosamente. La traducci\'on del s\'anscrito a otro idioma fue de valiosa ayuda, gracias. 

Un agradecimiento muy especial a todos mis colegas y amigos del Centro de Computaci\'on Gr\'afica. A la Prof. Omaira que siempre est\'a dispuesta a brindarme ayuda en todo lo que este a su alcance, as\'i como haberme dado la oportunidad de pertenecer a su centro de investigaci\'on, muchas gracias. A los profesores H\'ector, Rhadam\'es y Walter por sus valiosos consejos y recomendaciones que siempre son \'utiles y valiosas, adem\'as de compartir el d\'ia a d\'ia (exceptuando cuando est\'an de viaje). De igual forma, agradezco el constante intercambio de ideas con los estudiantes/tesistas del CCG.

Agradezco a mis amig@s que siempre est\'an interesados en mi desarrollo como profesional, con los cuales comparto momentos inolvidables. Especialmente a Nai, Mary, Dahyna, Sr. y Sra. De Jes\'us, Liliana, MaAng\'elica, Zuly y Edgar. 

Este trabajo ha sido financiado parcialmente por el Centro de Desarrollo Cient\'ifico y Human\'istico (CDCH) de la Universidad Central de Venezuela (UCV), a trav\'es de la beca-matr\'icula. Todas las im\'agenes han sido adquiridas con la ayuda especial del Dr. Carlos S\'anchez del Servicio de Traumatolog\'ia y Ortopedia del Hospital Universitario de Caracas. Quiero agradecer al Ing. Othman Falc\'on por haber propuesto el problema original, y por su apoyo incondicional en cada etapa en la cual pod\'ia hacerlo. El intercambio de ideas constante con el Dr. Carlos S\'anchez y su invaluable aporte no tiene precio, gracias partner.

Finalmente, gracias totales a la casa que vence las sombras por albergarme y darme siempre la oportunidad de crecer.





